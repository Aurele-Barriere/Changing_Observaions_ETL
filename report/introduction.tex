Epistemic Logics are well known to be a required formalism to describe and reason about knowledge in distributed systems. Distributed algorithms (also called \textit{protocols}) involve agents without complete knowledge of the state of the system. Therefore, ``Any logic of protocols must include as part of it a logic of knowledge'', as said by Ladner and Reif in~\cite{DBLP:conf/tark/LadnerR86}.
Such systems have been found to be useful for in many areas, including Game Theory and Artificial Intelligence.
A popular extension of Epistemic Logics is to combine them with Temporal Logics. Reasoning about the evolution of agents' knowledge inside a system becomes possible. For each of these logics, model-checking (deciding if a formula is true in a given model) is an important problem, as it allows to confront the model of a system to its specification.

In these settings, agents are usually given a fixed observation for the whole evolution. Observations describe an agent's point of view, to model imperfect information.
In this paper, to deal with dynamic changes of observations during the evolution in a system with imperfect information, we introduce a new logic, \ctlskd. To the best of our knowledge, this is the first time that such changes are studied. This logic includes branching-time temporal operators, epistemic operators, and a new one, $\D{o}$, to represent changes of observation. For instance, the formula $\D{o}\K\A\X p$ states that after changing to an observation $o$, the agent knows that, on the next step, the proposition $p$ holds.


This logic could be useful for any system where agents can change their observational power of the system. For instance, in a scenario where there exists different ``security levels'' where different levels have access to different information. With our logic, it becomes possible to express statements such as ``For an agent with initial observation $o_1$, there exists a point in time where, if the agent changes his observation to $o_2$, he knows whether or not some proposition holds'' ($\D{o_1}F(\D{o_2}(\K p\vee\K\neg p))$).
Another main motivation to define such a logic is the work that has been done on Strategy Logic with Imperfect Information~\cite{DBLP:conf/lics/BerthonMMRV17}, an extension of Strategy Logic~\cite{DBLP:journals/iandc/ChatterjeeHP10}. In this logic, agents can change observation when changing strategies.
Before investigating it in the full framework of Strategy Logic, the natural first step was to study the interactions of observation changes, knowledge and time in a simpler setting, without the strategic aspects.

In Section~\ref{sec:ctlskd}, we first define the logic \ctlskd. To begin, we only define it for single-agent synchronous perfect recall settings.
Then, in Section~\ref{sec:alternative}, we introduce an alternative, finitary semantics for the same formulas, that we later prove to be equivalent to the first one, and easier to model-check.
In Section~\ref{sec:mc}, we describe an algorithm to model-check a formula of \ctlskd.
Finally, we extend the logic for multi-agent settings in Section~\ref{sec:multi}.
